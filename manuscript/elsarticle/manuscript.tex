%%
%% Copyright 2007, 2008, 2009 Elsevier Ltd
%%
%% This file is part of the 'Elsarticle Bundle'.
%% ---------------------------------------------
%%
%% It may be distributed under the conditions of the LaTeX Project Public
%% License, either version 1.2 of this license or (at your option) any
%% later version.  The latest version of this license is in
%%    http://www.latex-project.org/lppl.txt
%% and version 1.2 or later is part of all distributions of LaTeX
%% version 1999/12/01 or later.
%%
%% The list of all files belonging to the 'Elsarticle Bundle' is
%% given in the file `manifest.txt'.
%%

%% Template article for Elsevier's document class `elsarticle'
%% with numbered style bibliographic references
%% SP 2008/03/01

%%\documentclass[doublespacing]{elsarticle}
\documentclass[preprint, 10pt]{elsarticle}

%% Use the option review to obtain double line spacing
%% \documentclass[authoryear,preprint,review,12pt]{elsarticle}

%% Use the options 1p,twocolumn; 3p; 3p,twocolumn; 5p; or 5p,twocolumn
%% for a journal layout:
%% \documentclass[final,1p,times]{elsarticle}
%% \documentclass[final,1p,times,twocolumn]{elsarticle}
%\documentclass[final,3p, times]{elsarticle}
%% \documentclass[final,3p,times,twocolumn]{elsarticle}
%% \documentclass[final,5p,times]{elsarticle}
%% \documentclass[final,5p,times,twocolumn]{elsarticle}

%% For including figures, graphicx.sty has been loaded in
%% elsarticle.cls. If you prefer to use the old commands
%% please give \usepackage{epsfig}

%% The amssymb package provides various useful mathematical symbols
\usepackage{amsmath}
\usepackage{amssymb}
\usepackage{mathtools}
%\usepackage[linesnumbered, ruled, vlined]{algorithm2e}
\usepackage{algorithm}% http://ctan.org/pkg/algorithms
\usepackage{algorithmicx}
\usepackage{algpseudocode}% http://ctan.org/pkg/algorithmicx
\renewcommand{\algorithmicrequire}{\textbf{Input:}}
\renewcommand{\algorithmicensure}{\textbf{Output:}}
\usepackage{enumerate}
\usepackage{multirow}
\usepackage{graphicx}
\usepackage{subfigure}
\usepackage{epstopdf}

\usepackage{booktabs}
\usepackage[dvipsnames]{xcolor}
\usepackage{tcolorbox}
\usepackage{soul}
%\tcbset{enhanced jigsaw, sharp corners,box align=center,boxrule=0pt}

\usepackage{lineno}

\usepackage{indentfirst}
%Bibliography
\usepackage{natbib}
\usepackage{hypernat}

\usepackage{lipsum} % Package to generate dummy text throughout this template

%\usepackage[sc]{mathpazo} % Use the Palatino font
\usepackage[T1]{fontenc} % Use 8-bit encoding that has 256 glyphs
%\linespread{2} % Line spacing - Palatino needs more space between lines
\usepackage{microtype} % Slightly tweak font spacing for aesthetics

\usepackage[hmarginratio=1:1,top=32mm,columnsep=20pt]{geometry} % Document margins
\usepackage{multicol} % Used for the two-column layout of the document
\usepackage[hang, small,labelfont=bf,up,textfont=it,up]{caption} % Custom captions under/above floats in tables or figures
\usepackage{booktabs} % Horizontal rules in tables
\usepackage{float} % Required for tables and figures in the multi-column environment - they need to be placed in specific locations with the [H] (e.g. \begin{table}[H])
\usepackage{hyperref} % For hyperlinks in the PDF

\usepackage{lettrine} % The lettrine is the first enlarged letter at the beginning of the text

\usepackage{amsthm,amsmath,amssymb}
\usepackage{mathrsfs}

%\usepackage{refcheck} % check the reference of equations, figures, tables and citations


\journal{Journal of Computational Physics}

\begin{document}
%	\linenumbers
	
\begin{frontmatter}

%% Title, authors and addresses

%% use the tnoteref command within \title for footnotes;
%% use the tnotetext command for theassociated footnote;
%% use the fnref command within \author or \address for footnotes;
%% use the fntext command for theassociated footnote;
%% use the corref command within \author for corresponding author footnotes;
%% use the cortext command for theassociated footnote;
%% use the ead command for the email address,
%% and the form \ead[url] for the home page:
%% \title{Title\tnoteref{label1}}
%% \tnotetext[label1]{}
%% \author{Name\corref{cor1}\fnref{label2}}
%% \ead{email address}
%% \ead[url]{home page}
%% \fntext[label2]{}
%% \cortext[cor1]{}
%% \address{Address\fnref{label3}}
%% \fntext[label3]{}

\title{Physics-informed machine learning of reduced-order model}

%% use optional labels to link authors explicitly to addresses:
%% \author[label1,label2]{}
%% \address[label1]{}
%% \address[label2]{}

\author[XJTU]{Wenqian Chen}
\ead{wenqianchen2016@gmail.com}
\author[EPFL]{Qian Wang\corref{cor1}}
\cortext[cor1]{Corresponding author.}
\ead{qian.wang@epfl.ch}
\author[EPFL]{Jan S. Hesthaven}
\ead{Jan.Hesthaven@epfl.ch}
\author[XJTU]{Chuhua Zhang}
\ead{chzhang@mail.xjtu.edu.cn}
\address[XJTU]{Department of Fluid Machinery and Engineering, School of Energy and Power Engineering, Xi'an Jiaotong University, Xi'an, Shaanxi, People's Republic of China}
\address[EPFL]{Chair of Computational Mathematics and Simulation Science, \'Ecole polytechnique f\'ed\'erale de Lausanne, 1015 Lausanne, Switzerland}

\begin{abstract}
%% Text of abstract
  \hspace{8pt} A physics-informed machine learning framework is developed for the reduced-order modeling of parametirzed steady-state partial differential equations (PDEs). In the offline stage, a reduced-order model is constructed by projecting the high-fidelity numerical model onto a reduced space, spanned by a set of basis functions that represent the main dynamics of the full-order model. A physics informed neural network (PINN), which approximates the mapping from the parameters to the reduced coefficients, is trained by minimizing the loss function as a sum of the residual of the reduced-order model and the matching error with the projection of high-fidelity snapshots onto the reduced space. In the online stage, given a new parameter location, the reduced coefficients are predicted by the physics-informed neural network. Numerical results demonstrate that the PINN can provide fast and accurate solutions to nonlinear problems.

\end{abstract}

\begin{keyword}
%% keywords here, in the form: keyword \sep keyword
   physics-informed machine learning \sep reduced-order modeling \sep nonlinear PDE

%% PACS codes here, in the form: \PACS code \sep code

%% MSC codes here, in the form: \MSC code \sep code
%% or \MSC[2008] code \sep code (2000 is the default)

\end{keyword}

\end{frontmatter}

%\linenumbers

%% main text
\section{Introduction}
Why we need reduced-order modeling.

Intrusive reduced-order model. POD-G is a classical choice. Pros and cons.

Intrusive reduced-order model. POD-NN is a recently developed one. Pros and cons. Especially the need of large amount of data for training.

We propose to use physics-informed machine learning to improve the reduced-order modeling by taking advantage of both physics and data. Then explain the idea.

The advantage of PINN over POD-G and POD-NN. Numerical results demonstrate the advantages.

The remainder of this paper is organized as follows.

\section{Projection-based reduced basis method}\label{sec:POD-G}
Consider a nonlinear dynamic system governed by partial differential equations(PDEs):
\begin{equation}
\begin{aligned}
\mathcal{N}\left(\phi\left(\mathbf{x}\right); \pmb{\mu}\right)=0&,  \qquad \mathbf{x} \in \Omega\left(\pmb{\mu}\right)  \\
\mathcal{B}\left(\phi\left(\mathbf{x}\right); \pmb{\mu}\right)=0&,  \qquad \mathbf{x} \in \partial \Omega\left(\pmb{\mu}\right)
\end{aligned}
\label{eq_GoverningEqs}
\end{equation}
where $\mathcal{N}$ is a general nonlinear differential operator, $\phi\left(\mathbf{x}\right)$ are field variables to be solved on Cartesian coordinates $\mathbf{x}$. $\mathcal{B}$ are operators defining boundary conditions for the boundary $\partial \Omega$ of the physical  domain $\Omega$. $\pmb{\mu} \in \mathcal{P}$ are specific parameters characterizing the nonlinear dynamic system as well as the physical domain $\Omega$, where $\mathcal{P}$ is the parameter space.

As is always the case for reduced-order methods, a dynamic system is approximated by finding the its projection in a space spanned by a few well-chosen basis vectors. These basis are generated from a suitable linear combination of some precomputed high-fidelity approximations, termed as snapshots. To ensure the compatibility among snapshots, the variable physical domain has to be addressed. To this end, the variable physical domains will be mapped into a fixed computational domain with the same collation of degree of freedoms. This can be implemented by a invertible problem-dependent mapping $\mathcal{X} : \mathbf{x} \in \Omega\left(\pmb{\mu}\right) \to \pmb{ \xi} \in \widetilde \Omega$, which reads
\begin{equation}
\pmb{ \xi} = \mathcal{X} \left( \mathbf{x}; \pmb{\mu} \right),
\label{eq_Transform}
\end{equation}
and thus the governing equations (\ref{eq_GoverningEqs}) are recast as follows:
\begin{equation}
\begin{aligned}
\mathcal{N}\left(\phi\left( \mathcal{X}^{-1} \left( \pmb{ \xi}; \pmb{\mu} \right) \right); \pmb{\mu}\right)
&:=\widetilde {\mathcal{N}} \left(\phi\left( \pmb{ \xi} \right); \pmb{\mu}\right)
=0,  \qquad \pmb{ \xi} \in \widetilde \Omega
\\
\mathcal{B}\left(\phi\left(\mathcal{X}^{-1} \left( \pmb{ \xi}; \pmb{\mu} \right)\right); \pmb{\mu}\right)
&:=\widetilde {\mathcal{B}} \left(\phi\left( \pmb{ \xi} \right); \pmb{\mu}\right)
=0,  \qquad \pmb{ \xi} \in \partial \widetilde \Omega
\end{aligned}
\label{eq_GoverningEqsForXi}
\end{equation}
where $\widetilde \Omega$ is the computational domain, $\widetilde {\mathcal{B}}$ and $\widetilde {\mathcal{N}}$ are derived operators in computational space resulting from transformation Eq. (\ref{eq_Transform}).

%In this work, we are interested in finding $\phi\left(\pmb{\xi}\right)$ for a given $\pmb{\mu} \in \mathbb{P}$


\subsection{Full-order model}
To simulate the nonlinear dynamic system in Eqs. (\ref{eq_GoverningEqsForXi}), the including spatial derivatives are always first discretized by a suitable high-fidelity (HF) method (such as the pseudospectral method, spectral difference method, etc.). To achieve a satisfactory numerical accuracy, a fine mesh is employed with a large number of degrees of freedom (DOFs), generally including $N$ interior DOFs and $N_B$ boundary DOFs. Finally, the resulting discretized equations for  Eqs. (\ref{eq_GoverningEqsForXi}) are solved with a suitable explicit/implicit solver.

Here we denote the discrete solutions of Eq ( \ref{eq_GoverningEqsForXi}) in vector form $\pmb{\phi} _h \in \mathbf{R}^{N  }$ and $\pmb{\phi} _h^{B} \in \mathbf{R}^{N_B}$ for parameter $\pmb{\mu}  \in \mathcal{P}$.
Given a suitable HF method, we have the governing Eq. (\ref{eq_GoverningEqsForXi}) in discrete form
\begin{align}
\mathbf{L}_{\widetilde {\mathcal{N}}}\pmb{\phi}_h +g_{\widetilde {\mathcal{N}}}(\pmb{\phi}_h, \pmb{\phi}_h^{B})
+\mathbf{C}_{\widetilde {\mathcal{N}}}
= 0
\label{eq_DiscreteEqs}
\\
\mathbf{L}_{\widetilde {\mathcal{B}}} \pmb{\phi}_h + \mathbf{L}_{\widetilde {\mathcal{B}}}^{B} \pmb{\phi}_h^{B}
+\mathbf{C}_{\widetilde {\mathcal{B}}}
= 0
\label{eq_DiscreteBoundary}
\end{align}
where $\mathbf{L}_{\widetilde {\mathcal{N}}} \in \mathbf{R}^{N \times N}$, $\mathbf{L}_{\widetilde {\mathcal{B}}} \in \mathbf{R}^{N_B \times N}$ and $\mathbf{L}_{\widetilde {\mathcal{B}}}^{B} \in \mathbf{R}^{N_B \times N_B}$ represent matrixes derived from linear parts of operators $\widetilde {\mathcal{N}}$ and $\widetilde {\mathcal{B}}$.
$\mathbf{C}_{\widetilde {\mathcal{N}}} \in \mathbf{R}^{N}$ and $\mathbf{C}_{\widetilde {\mathcal{B}}} \in \mathbf{R}^{N_B}$ are constant vectors independent of $\pmb{\mu}$.
$g_{\widetilde {\mathcal{N}}}: \mathbf{R}^N \times \mathbf{R}^{N_B} \mapsto \mathbf{R}^N$ is a nonlinear function derived from nonlinear part of operator $\widetilde {\mathcal{N}}$.

For the treatment of boundary conditions, it can classified into two categories, namely weakly enforced and strongly enforced boundary conditions.
As for weakly enforced boundary conditions, no DOFs are required to deploy on the boundaries, i.e. $N_B=0$, implying a vanishing of $\pmb{\phi} _h^{B}$ and also Eq. (\ref{eq_DiscreteBoundary}). As for strongly enforced boundary conditions, a suitable collocation of DOFs on boundaries is necessary. Note that the commonly used boundary conditions always linear ones, such as Dirictlet, Neumann and Robin conditions, and thus $\mathcal{B}$ are always linear operators. Without loss of generality, we restrict ourselves that the mapping in Eq. (\ref{eq_Transform}) will retain the linearity of boundary condition operator $\widetilde {\mathcal{B}}$ in computational space. Therefore, discretization of the boundary conditions result in a linear system of equations, i.e., Eq. (\ref{eq_DiscreteBoundary}).

In what remains, the Chebyshev pseudospectral method, characterized by its outstanding accuracy, is chosen for discretizing Eqs. (\ref{eq_GoverningEqsForXi}),  which will be detailed in the following subsection.

\subsubsection{Chebyshev pseudospectral method}
In the Chebyshev pseudospectral method, the $d$-dimensional variable physical domain of  will be mapped into a unit regular computational domain $\widetilde \Omega =[-1,1]^d$,  each dimension discretized by  $N_p+1$ Chebyshev-Gauss-Lobatto points,
\begin{equation}
\xi_i=\cos \left( \frac{\pi i}{N_p} \right), \quad 0 \le i \le {N_p}
\label{eq_CollocationPoints}
\end{equation}
Spatial derivatives in operators $\widetilde {\mathcal{N}}$ and $\widetilde {\mathcal{B}}$ are approximated with a matrix-vector multiplication in each dimension. In each dimension, the derivatives are approximated as follows:
\begin{equation}
\left. { \frac{\partial^s    \phi}{{\partial \xi^s}  }} \right|_{\xi _i }
 = \mathbf{D}^s \pmb{\phi}
\end{equation}
where $\pmb{\phi}=\{\phi _i\}_{i=0}^{N_p}$, $s$ is the order of derivative and $\mathbf{D}^s$ is the $s$th-order difference matrix of size ${(N_p+1)\times (N_p+1)}$ with entries defined by
\begin{equation}
D^0_{i,j} = \delta_{ij} \qquad 0 \le i,j \le {N_p},
\label{eq_D0}
\end{equation}
\begin{equation}
\left\{ \begin{aligned}
&D_{i,j}^1 = \frac{{{B_i}}}{{{B_j}}}\frac{{{{( - 1)}^{i + n}}}}{{2\sin \left( {\frac{{(i + j)\pi }}{{2{N_x}}}} \right)\sin \left( {\frac{{(j - i)\pi }}{{2{N_p}}}} \right)}} \qquad 0 \le i,j \le {N_p},i \ne j\\
&D_{i,i}^1 =  - \sum\limits_{j = 0,j \ne i}^{{N_p}} {D_{i,j}^1} \quad 1 \le i \le {N_p} - 1\\
&D_{0,0}^1 =  - D_{{N_p},{N_p}}^1 =  - \frac{{2{N_p}^2 + 1}}{6}{\rm{                       }}\\
&{B_i} = \left\{
                 \begin{array}{*{20}{l}}
                      2 \qquad i=0, N_p \\
                      1 \qquad 1 \le i \le {N_p-1}
                 \end{array}
                 \right.
\end{aligned} \right. ,
\label{eq_D1}
\end{equation}
\begin{equation}
D^s_{i,j} = D^{1}_{i,k}D^{s-1}_{k,j} \qquad 0 \le i,j,k \le {N_p}.
\label{eq_Ds}
\end{equation}

For the following simulations of flow problems, the well known $IP_N-IP_{N-2}$ method is adopted to prevent spurious pressure mode from containing the flow fields. In the $IP_N-IP_{N-2}$ method, pressure derivative is approximated without boundary points, and thus pressure is approximated with polynomial of two order lower than all other field variables, such as velocity and temperature. All other variables are discretized with Eqs. (\ref{eq_D0})-(\ref{eq_Ds}) except that the first-order difference matrix $\mathbf{D}^{1}$ of pressure is replaced with $\hat {\mathbf{D}}^1$ as follows
\begin{equation}
\left\{ \begin{aligned}
&\hat D_{i,j}^1 = 0                              & i=0,N_p \text{  or } j=0, N_p \\
&\hat D_{i,i}^1 = \frac{3\xi_i}{2(1-\xi_i^{2})}  & 1 \le i \le N_p-1\\
&\hat D_{i,j}^1 = \frac{(-1)^{i+j}(1-\xi_j^{2})}
{2(1-\xi_i^{2})(\xi_i-\xi_j)} \qquad             & 1 \le i \ne j \le N_p-1\\
\end{aligned} \right. ,
\label{eq_D1}
\end{equation}


For more details of the Chebyshev pseudospectral method, we refer the reader to the reference.


\subsection{Reduced-order model}
\label{sec_ROM}
As for solving the problem in Eq. (\ref{eq_GoverningEqsForXi}), full-order model is always suffering from large computational cost since the required degree of freedom $N$ is always very large.  Therefore, we are interested in replacing the full-order model with a well-posed  reduced-order model to represent main dynamic of the system. The full-order model is termed as reducible when Eq. (\ref{eq_FullOrderModel}) can be well approximated with $m$-dimensional subspace $\mathcal{V}=\mathbf{R}^{N \times m}$. The subspace is spanned by a suitable set of basis vectors $\{\mathbf{V}_i \in \mathbf{R}^{N} \}_{i=1}^m$.  Note that the the size of problem will be largely reduced, only if $m \ll N$. The full-order solution of the interior DOFs $\pmb{\phi} _h$ can be projected into the subspace $\mathcal{V}$
\begin{equation}
\begin{aligned}
\pmb{\phi} _h  &   =   \mathbf{V} \pmb{\alpha} + \tilde{\phi} + \pmb{\epsilon}
               &\approx \mathbf{V} \pmb{\alpha} + \tilde{\phi}
\end{aligned}
\label{eq_projection}
\end{equation}
where $\mathbf{V} \in \mathbf{R}^{N\times m}$ is a matrix with the basis vectors as its columns, $\pmb{\alpha}$ is   the coefficients of the basis, $\pmb{\epsilon}$ is the projection error, $\mathbf{I} \in \mathbf{R}^{N}$ is filled with 1.  $\tilde{\phi} \in \mathbf{R}$ is independent of $\pmb{\mu}$, set for filtering a constant value to avoid its domination on the basis, and usually it can be set as the average. The boundary conditions say that the interior DOFs have a linear relationship with interior DOFs. Thus, we can enforce the boundary conditions without any reduction. The procedure keeps the reduced-order model of high accuracy at bounaries and avoid shifting of boundary values. According to Eq.(\ref{eq_DiscreteBoundary}), the boundary DOFs can be denoted as follows
\begin{equation}
\pmb{\phi}_h^{B} = -( \mathbf{L}_{\widetilde {\mathcal{B}}}^{B} )^{-1}
\left(
\mathbf{L}_{\widetilde {\mathcal{B}}} \pmb{\phi}_h +
\mathbf{C}_{\widetilde {\mathcal{B}}}
\right)
\label{eq_DerivePhiB}
\end{equation}

Substituting Eq. (\ref{eq_projection}) and (\ref{eq_DerivePhiB}) into Eq. (\ref{eq_DiscreteEqs}), we have the overdetermined system for interior
DOFs
\begin{equation}
\mathbf{L}_{\widetilde {\mathcal{N}}} \mathbf{V} \pmb{\alpha}
+\mathbf{g}_{\widetilde {\mathcal{N}}}
\left( \mathbf{V} \pmb{\alpha} + \tilde{\phi} + \pmb{\epsilon}, -\left( \mathbf{L}_{\widetilde {\mathcal{B}}}^{B} \right)^{-1}
\left(
\mathbf{L}_{\widetilde {\mathcal{B}}} \mathbf{V} \pmb{\alpha} +
\mathbf{C}_{\widetilde {\mathcal{B}}} + \mathbf{L}_{\widetilde {\mathcal{B}}} \mathbf{I} \tilde{\phi} + \mathbf{L}_{\widetilde {\mathcal{B}}} \pmb{\epsilon}
\right) \right)
+\mathbf{L}_{\widetilde {\mathcal{N}}} \mathbf{I} \tilde{\phi}
+\mathbf{L}_{\widetilde {\mathcal{N}}} \pmb{\epsilon}
+ \mathbf{C}_{\widetilde {\mathcal{N}}}
= 0
\label{eq_Overdetermined0}
\end{equation}
For the sake of clarity, Eq. (\ref{eq_Overdetermined0}) can be rearranged in the simplified form
\begin{equation}
 \mathbf{L} \mathbf{V} \pmb{\alpha}
+ g         \left( \mathbf{V} \pmb{\alpha} \right)
+\mathbf{C}
= \tilde{ \pmb{\epsilon} }
\label{eq_Overdetermined1}
\end{equation}
where $\mathbf{L} \in \mathbf{R}^{N \times N}$, $\mathbf{C} \in \mathbf{R}^{N}$ and $g: \mathbf{R}^N \mapsto \mathbf{R}^N$ is the matrix, vector and function derived from Eq. (\ref{eq_Overdetermined0}), respectively.
The quantity $\tilde{ \pmb{\epsilon} }$ represent the residual resulting from the projection error $\pmb{\epsilon}$.
In the framework of a Petrov-Galerkin projection, if we consider a basis $\mathbf{W} \in \mathbf{R}^{N \times m}$ that is orthogonal to the residual,  the overdetermined system is reduced to a system of $m$ equations
\begin{equation}
\mathbf{W}^T
\left(
 \mathbf{L} \mathbf{V} \pmb{\alpha}
+\mathbf{g} \left( \mathbf{V} \pmb{\alpha} \right)
+\mathbf{C}
\right)
=0
\label{eq_Petrov}
\end{equation}
In this work, we restrict ourself to Galerkin framework, namely $\mathbf{W}$=$\mathbf{V}$.
Thus, we have the reduced-order model as follows:
\begin{equation}
\mathbf{V}^T
\left(
 \mathbf{L} \mathbf{V} \pmb{\alpha}
+ g         \left( \mathbf{V} \pmb{\alpha} \right)
+\mathbf{C}
\right)
=0
\label{eq_Galerkin}
\end{equation}

As discussed above, the key point of the reduced-order model is to get a suitable basis. In the literature, there exists many methods to achieve it. Such popular methods includes, but not limited to, the Proper Orthogonal Decomposition (POD), the Proper Generalized Decomposition, the Piecewise Tangential Interpolation, the Matrix Interpolation, the Loewner framework and several greedy-based approaches for the identification of sub-optimal subspaces. Among these methods, POD, perhaps the most widely used technique, is described and employed in the following.


\subsubsection{POD reduced basis}
The Proper Orthogonal Decomposition (POD) is one of the most widely used techniques to compress data and extract fundamental information by building a series of orthogonal basis
with decreasing energy distribution.
Let $\mathcal{P}_M=\{\pmb{\mu}_1, \pmb{\mu}_2,...,\pmb{\mu}_M\} \subset \mathcal{P}$ be a discrete and finite set of $M$ parameters generated with a suitable sampling method, and $\pmb {\Phi}_{\mathcal{P}_M}=\{\pmb{\phi}_h(\pmb{\mu}_1), \pmb{\phi}_h(\pmb{\mu}_2),...,\pmb{\phi}_h(\pmb{\mu}_M)\}  \in \mathbf{R}^{N \times M}$ be the corresponding snapshot matrix obtained by full-order method. Consider that snapshot matrix $\pmb {\Phi}_{\mathcal{P}_M}$ can represent the underlying physics dynamics of the problem. The idea behind POD technique is to find a suitable orthogonal basis matrix to minimise the problem defined as follows:
\begin{equation}
\begin{aligned}
&\mathop {\min}\limits_{\mathbf{V} \in \mathbf{R}^{N\times m}}
\left\| {\pmb {\Phi}_{\mathcal{P}_M} - \mathbf{V}\mathbf{V}^T \pmb {\Phi}_{\mathcal{P}_M}} \right\|_F, \\
&s.t. \qquad \mathbf{V}^T\mathbf{V} = \mathbf{E},
\end{aligned}
\label{eq_PODProblem}
\end{equation}
where $\mathbf{E}$ is the identify matrix of size $M$ and $\left\| \cdot \right\|_F$ is the Frobenius norm.
Galerkin is a popular choice of projection.
According to the Eckart-Young theorem, the solution of Eq. (\ref{eq_PODProblem}) is exactly  the first $m$th left singular vectors of the matrix $\pmb {\Phi}_{\mathcal{P}_M}$, derived from singular value decomposition which reads
\begin{equation}
\pmb {\Phi}_{\mathcal{P}_M} = \mathbf{U}_S \pmb{\Sigma} _S \mathbf{V}_S, \qquad \pmb{\Sigma} _S = \text{diag} \left(  \sigma _i  \right)
\end{equation}
where  the singular values $\sigma _i$, $1 \le i \le \min(N,M)$ are sorted in a decreasing order. Choosing the first $m$th columns of
 $\mathbf{U}_S$, we have an error estimation of Eq. (\ref{eq_PODProblem}) as
 \begin{equation}
 \mathop {\min}\limits_{\mathbf{V} \in \mathbf{R}^{N\times m}}
\left\| {\mathbf{S} - \mathbf{V}\mathbf{V}^T\mathbf{S}} \right\|_{F}^{2}
=\sum_{i=m+1}^{\min(N,M)} \sigma _{i}^{2}
\label{eq_PODError}
 \end{equation}
In Eq. (\ref{eq_PODError}), it is shown that the error is exactly made up with  the squares of the neglected singular values. That is to say, with a suitable $m$, we can approximate the snapshot matrix $\pmb {\Phi}_{\mathcal{P}_M}$ at an arbitrary accuracy.  Luckily, most problems exhibit an exponentially decaying series of singular values, and thus we can chose a very small value of $m$ to approximate the problems with a good accuracy.

\subsubsection{Further reduction in online cost}
Although we have the reduced-order model defined in Eq. (\ref{eq_Galerkin}), the computational cost is still very expensive, scaling with the original size of teh full-order model. For online stage, a further reduction of  computational cost is required. For linear parts in Eq. (\ref{eq_Galerkin}), they can be treated as follows
\begin{equation}
\begin{aligned}
\mathbf{V}^T \mathbf{L} \mathbf{V} \pmb{\alpha}  &= \tilde {\mathbf{L}} \pmb{\alpha} \\
\mathbf{V}^T \mathbf{C}                          &= \tilde {\mathbf{C}}
\end{aligned}
\label{eq_ReductionLinear}
\end{equation}
where $\tilde {\mathbf{L}} \in \mathbf{R}^{m \times m}$ and $\tilde {\mathbf{C}} \in \mathbf{R}^{m} $ can only be computed once with the knowledge of basis matrix $\mathbf{V}$.
Therefore, the online calculation of linear parts scales with $m$. However, the drawback of the reduced-order model is that the cost of exact evaluation of nonlinear part $\mathbf{V}^T g \left( \mathbf{V} \pmb{\alpha} \right)$  in Eq. (\ref{eq_Galerkin}) scales with $N$, namely the size of full-order model. To tackle this defect, several hyper-reduction methods have been developed in the last decades to enable significant speedups for nonlinear part. These methods tries to find the optimal tradeoff between accuracy and efficiency. Such methods include but not limited to the Empirical Interpolation method (EIM), its
discrete variant (DEIM), Gappy-POD and Missing Point Estimation (MPE). In the present work, we restrict ourselves to quadratic nonlinearity, which can be exactly transformed to quadratic form, and thus hyper-reduction is avoid. Consider a common quadratic nonlinearity
\begin{equation}
g \left( \mathbf{V} \pmb{\alpha} \right)
=
\left( \mathbf{V} \pmb{\alpha} \right) \otimes \left( \mathbf{V} \pmb{\alpha} \right)
\label{eq_gQuadratic}
\end{equation}
where $\otimes$ denotes element-wise multiplication operator. Substituting Eq. (\ref{eq_gQuadratic}) in to Eq. (\ref{eq_Galerkin}), the nonlinear part can be transformed as
\begin{equation}
\begin{aligned}
\mathbf{V}^T
g \left( \mathbf{V} \pmb{\alpha} \right)
&=
\mathbf{V}^T
\left(
\left( \mathbf{V} \pmb{\alpha} \right) \otimes \left( \mathbf{V} \pmb{\alpha} \right)
\right)
&= \sum_{k=0}^{m} { \left( \pmb{\alpha}^T \mathbf{A}^k \pmb{\alpha} \right) \mathbf{E}_k }, \qquad 0 \le k \le m
\end{aligned}
\label{eq_ReductionNonlinear}
\end{equation}
where $\mathbf{E}_k \in \mathbf{R}^{m}$ is $k$th column of the unit matrix $\mathbf{E}$, $\mathbf{A}^k \in \mathbf{R}^{m \times m}$ is calculated as
\begin{equation}
\mathbf{A}_{i,j}^k = \mathbf{V}_k \cdot \left( \mathbf{V}_i \otimes \mathbf{V}_j \right), \qquad 0 \le i,j,k \le m
\end{equation}
where $\mathbf{V}_i$, $\mathbf{V}_j$ and $\mathbf{V}_k$ are the $i$th, $j$th and $k$th columns of $\mathbf{V}$, respectively.

\section{Physics-informed machine learning of reduced-order model}
This section presents serval methods, classified as intrusive and non-intrusive methods,to find a good mapping from parameter space to the RB coefficients, namely the projections of parameterized PDE solutions. We term it as a non-intrusive method if the reduced-order model won't need to be built, otherwise, it is a intrusive method. As for non-intrusive methods in reference, the underlying idea is resorting to interpolation, such as cubic spline method based on a tensor-product grid in parameter space and building an artificial neural network (ANN) with a labeled data. These non-intrusive methods can achieve a good approximation and very little online cost, but the drawback is their highly relying on the size of labeled data, namely the number of evaluations of full-order model. According to Jan, artificial neural network has a great reduction in the eagerness of more labeled data compared with the cubic spline method. Even so, the required size of labeled data is prohibitively larger than the size of snapshots for building a satisfactory reduced basis. An good alternate is the intrusive method, where the reduced-order model with a suitable method, such as Newton-like method. However, it is also a tough work to find the right solution especially for a larger $m$. To this end, we combine ANN and the reduced-order model together together, which is also referred as  physics-informed neural network (PINN).

In the next subsections, we first briefly introduce the basis idea of artificial neural networks. Then we will present three training methods to train the network, namely POD-NN proposed by Jan, PODD-PINN1 and POD-PINN2 developed in this work, for finding the mapping from parameter space to the RB coefficients.

%They enjoy the similar network architecture, but take different definitions of loss function and training methods. Before the training of the networks, some dataset need to prepared first. Now we have $M$ snapshots for the set $\mathcal{P}_M$, generate and choose the first $m$th POD basis vectors.

\subsection{Artificial neural network (ANN) }
For an arbitrary target function, ANN is a powerful nonlinear approximation constituted of a linear combination of some nonlinear approximators. Generally, ANN contains $L$ hidden layers besides input and output layers,  the $i$th layer equipped with $n_l$ neurons, where $l=0,1,..L,L+1$ denotes the input layer, 1st hidden layer,..., $L$th hidden layer and the output layer, respectively. ANN is a nonlinear function $\mathcal{O}: \mathbf{R}^{n_0} \mapsto \mathbf{R}^{n_{L+1}}$ of inputs $\mathbf{x}$, comprising a concise recursion
\begin{equation}
\left \{
\begin{aligned}
&\mathcal{O}^{0}   = \mathbf{x} \\
&\mathcal{O}^{l}   = \sigma ^{l} \left( \mathbf{W}^{l} \mathcal{O}^{l-1} + \mathbf{b}^{l}
\right)
\qquad 1 \le l \le L+1
\end{aligned}
\right.
\end{equation}
where $\mathcal{O}^{l} \in \mathbf{R}^{n_l}$ is the output of $l$th layer, $n_0$ is the input dimesion and $n_{L+1}$ is the output dimension, $\mathbf{W}^{l} \in \mathbf{R}^{n_{l} \times n_{l-1}}$ is the weights, $\mathbf{b}^{l} \in \mathbf{R}^{n_l}$ is the biases and $\sigma ^{l}$ is a element-wise activation function. Usually, the activation function of the output layer is just set as the identification function, namely $\sigma ^{L+1}\left(\mathbf{x}\right)=\mathbf{x}$, while the activation function of all hidden layers are set as a same nonlinear function.

In this work, we simply set the activation function as Swish function $ x \cdot \text{sigmoid}(x)$ and use the same number of neurons in each hidden layers $n_1=n_2=,...,=n_L=n_H$ , if not stated otherwise. For a specific ANN architecture ($L$ the number of hidden layers and $n_H$ the number of neurons), the weights and biases of ANN are trained to minimize the discrepancy between ANN outputs and targets for the given inputs, where the discrepancy is a scalar function referred to the loss function. Thus the training of ANN is essentially an single-objective minimization problem
\begin{equation}
\mathop {\arg \;\min }\limits_{{\mathbf{W,b}}} \;
loss\left( {\mathcal{O}\left(\mathbf{x}\right),\mathcal{O}_\text{target} \left(\mathbf{x}\right)} \right)
.
\end{equation}
However, optimizing the minimization problem is not trivial. As for approximating a complex nonlinear target function with high dimensional inputs/outputs, always ANN of large depth and width is necessary, making the size of independent variables in $\mathbf{W,b}$ very big. Thus many numerical issues will emerge, among which local minima and overfitting traps are the most common cases. Luckily, many training techniques have been developed in last decades, such as Mini-Batch method to relive local minima trap and regulation method to relieve overfitting trap.

\subsection{Training the networks}
%To get a good mapping form parameter $\pmb{\mu} \in \mathcal{P}$ to the projection of snapshot $\pmb{\phi}_h \in \mathbf{R}^{N}$ in reduced sapce, a ANN is built and trained based on reduced basis.

Some preparation  need to be made before training the networks. Recall that the snapshot set $\pmb {\Phi}_{\mathcal{P}_M}=\{\pmb{\phi}_h(\pmb{\mu}_1), \pmb{\phi}_h(\pmb{\mu}_2),...,\pmb{\phi}_h(\pmb{\mu}_M)\}  \in \mathbf{R}^{N \times M}$ Corresponding with the parameter set $\mathcal{P}_M=\{\pmb{\mu}_1, \pmb{\mu}_2,...,\pmb{\mu}_M\} \subset \mathcal{P}$ is employed to extract the reduced basis $\mathbf{V}$ , according to Section \ref{sec_ROM}. Due to the orthogonality of $\mathbf{V}$, the projection coefficient of each snapshot on the reduced space is derived by left multiplying Eq. (\ref{eq_projection}) with $\mathbf{V}^T$
\begin{equation}
\begin{aligned}
\pmb{\alpha}  &   =    \mathbf{V}^T (\pmb{\phi}_h -\tilde{\phi} - \pmb{\epsilon}) \\
              &\approx \mathbf{V}^T (\pmb{\phi}_h -\tilde{\phi})
\end{aligned}
\end{equation}

All the parameters in the parameter space along with the projections of their corresponding snapshots, denoted by $\mathcal{D}_{POD}=\left\{\left(\pmb{\mu}_i, \left. {\pmb{\alpha}} \right|_{\pmb{\mu}=\pmb{\mu}_i}
\right) \right\}_{i=1}^{M}$,  will be a good collection of input-output data set for training an ANN. As the projection is the best approximation to the snapshot with only the projection error, the data set $\mathcal{D}_{POD}$ is the best choice for training an ANN.

As ANN is sensitive to the difference in dimensional scale of inputs/outputs,
the data set $\mathcal{D}_{POD}$ need to be further normalized before feeding it into the ANN. For inputs, namely $\pmb{\mu} \in \mathcal{P}$, as the range of the parameter space is always acknowledged, the minimal and maximal of the parameter space can be utilized to scale the inputs to $[-1,1]^{n_0}$, which is formulated as follows
\begin{equation}
\tilde{\pmb{\mu}}(\pmb{\mu}) =
\frac
{{\pmb{\mu}}- ( {\pmb{\mu}}_{\max} +{\pmb{\mu}}_{\min} )/2}
{             ( {\pmb{\mu}}_{\max} -{\pmb{\mu}}_{\min} )/2}
\end{equation}
As for the outputs, there is not an prior range of the output, however. We turn to a statistical method do the normalization with an assumption that the outputs satisfy the Gaussian distribution. To this end, the standard derivation $\Theta$ and mean $\overline {\alpha}$ are calculated, and then the outputs are normalized as follows
\begin{equation}
\tilde{\pmb{\alpha}}(\pmb{\alpha}) =
\frac
{{\pmb{\alpha}}- \overline {\pmb{\alpha}}}
{     \pmb{\Theta}         }
\end{equation}
Thus we wrap the normalisation methods into the ANN as follows
\begin{equation}
\tilde{\mathcal{O}}\left(\pmb{\mu}; \mathbf{W,b} \right) = \tilde{\pmb{\alpha}} \left(
{\mathcal{O}}\left(\mathbf{ \tilde{\pmb{\mu}}(\pmb{\mu}) }; \mathbf{W,b}\right)
\right)
\end{equation}


\subsubsection{PDNN:POD-based projection-driven neural network}
According to Jan, the data set $\mathcal{D}_{\rm{POD}}$ is used directly used to train the network to avoid building the reduced-order model. The data set $\mathcal{D}_{\rm{POD}}$ is randomly split into two parts with an ratio: train set $\mathcal{D}_{\rm{POD}}^{tr}$ and test set $\mathcal{D}_{\rm{POD}}^{te}$. The famous Adam optimizer is employed to train the network. The ratio between the size of $\mathcal{D}_{\rm{POD}}^{tr}$ and $\mathcal{D}_{\rm{POD}}$  is set as 0.7. The loss function is defined as the mean square error between network outputs and targets scaled with the standard derivation $\Theta$, namely
\begin{equation}
loss_{\rm{NN}} = \frac{1}{N_{\cal D}}
{{\sum\limits_{\pmb\mu ,\pmb\alpha  \in {\cal D}} {{{\left\| {\frac{\left( {\widetilde {\cal O}\left( {\pmb\mu ;{\bf{W}},{\bf{b}}} \right) - \pmb{\alpha} } \right)} {\pmb{\Theta}} } \right\|}_2}} }}
\end{equation}
where $\cal D$ is the chosen data set. The train set is used to

In real applications, we are not expected to generate too many snapshots, since the full-order model is always prohibitively expensive. Thus the data set size $\mathcal{D}_{\rm{POD}}$ is always to very small, so the training of the network will get trapped into overfitting. To prevent overfitting, first the $L_2$ regulation technique employed by penalizing the weights $\bf{W}$ in the loss function
\begin{equation}
\tilde{loss}_{\rm{NN}}
 = loss_{\rm{NN}}
  + \eta {\left\| { \mathbf{W} } \right\|}
\end{equation}
Besides, we adopt the early stopping criterion: training will be immediately stopped only if the validation error keeps increasing over $K_{early}$ epoches. In this work, $K_{early}=6$ is employed.

\subsubsection{PINN: POD-based physics-driven neural network}
Due to the limited data, it is always difficult to train a good network. Luckily, we have the reduced-order model, an another choice for training an ANN. Given the reduced model, ANN output need not  explicitly approximate some given targets, but just satisfying reduced-order model. In this way, no target is needed, and thus we have unlimited training data. By Latin hypercube sampling in parameter space, we generate the data set $\mathcal{D}_{\rm{PINN}}=\{ \pmb{\mu}_i\}_{i=1}^{M_{\rm{PINN}}}$.
The loss function is defined as follows
\begin{equation}
{loss}_{\rm{PINN}} = \frac{1}{N_{\cal D}}
{{\sum\limits_{\pmb\mu  \in {\cal D}}
{ ROM \left(\widetilde {\cal O}\left( {\pmb\mu ;{\bf{W}},{\bf{b}}} \right) \right)}
}}
\end{equation}
where $ROM(\pmb{\alpha})=\mathbf{V}^T
\left(
 \mathbf{L} \mathbf{V} \pmb{\alpha}
+ g         \left( \mathbf{V} \pmb{\alpha} \right)
+\mathbf{C}
\right)$
is the reduced-order model defined in Eq. (\ref{eq_Galerkin}).
As we have unlimited training data, there is np problem of overfitting and thus no need of  validation data set.  The bottleneck of training the POD-PINN is how to handle the local minima trap. Here, we adopt the Mini-Batch training method, where the data set are shuffled and randomly in to serval non-overlap subsets, in each epoch. With each subset, the weights and biases will be updated by optimizer. the Mini-Batch training method can effectively avoid local minima trap and has enjoyed a great number of applications.

\subsubsection{PRNN: POD-based physics-reinforced neural network with labeled data}
POD-PINN1 seems an perfect potential in prediction RB coefficients. But the numerical experiments show that the the accuracy of the POD-PINN1 sometimes is not not good, even if the the corresponding loss drops down to a very low level. The underlying reason comes from that the large scale difference of different RB modes.
First, the RB mode with smaller index will be more dominant to the residual of the reduced-order model. Second, the training of network cannot work like a deterministic nonlinear solver, and it can only reduce the loss to a moderate level. Thus, the reduced-order model cannot be well-solved thorough training network. As a result, the optimizer tends to neglect the relatively unimportant modes, namely the modes with a larger index.

To address this new problem, we introduce the projection RB coefficients into the loss function POD-PINN1. Although the projection RB coefficients are always not the solutions of the reduced-order model, the projection RB coefficients are more close to the snapshots and thus full-order model. Besides, the projection RB coefficients can be calculated directly, and it has no problem of domination for different modes. Therefore, the projection RB coefficients can be used to provide more physical information into the network training. Here we name this method as POD-PINN2. The loss function is defined as the weighted sum of those of POD-NN and POD-PINN, namely
\begin{equation}
{loss}_{\rm{PINN2}} =
\left. {{loss}_{\rm{NN}}} \right|_{\cal D = {\cal D}_{\rm{POD}} }
+
w \times \left. {{loss}_{\rm{PINN}}} \right|_{\cal D = {\cal D}_{\rm{PINN}} }
\end{equation}
where $w$ is a specific weight balancing projection and reduced-order model.
when training the network, the two data sets are treated separately. As for projection data set ${\cal D}_{\rm{POD}}$, it more accurate but its size is very small, so we continue to use full batch. As for residual data set ${\cal D}_{\rm{PINN}}$, it is less accurate but its size is very large, and we use mini-batch method for each updating of the network.

\section{Numerical results}
In this section, we will discuss the results obtained  via the PS, POD-G, POD-NN, POD-PINN1, POD-PINN2 methods applied to parameterized nonlinear PDEs for the one-dimensional Burges' equation, two-dimensional lid driven flow and two-dimensional natural convection flow.

To assess the online accuracy of the present method, the following metrics are defined in the sense of mean squared error.
\begin{enumerate}[(1)]
\item the average relative error of  projections
\begin{equation}
\varepsilon_{\rm{Proj}}
=\frac{1}{N_{{\cal D}_{te}}}
\sum\limits_{\pmb\mu ,\pmb{\phi}_h  \in {{\cal D}_{te}}}{
\frac{\left\| \pmb{\phi}_h - \tilde{\phi} -\mathbf{V}\mathbf{V}^T \left( \pmb{\phi}_h - \tilde{\phi} \right)  \right\|}
{\left\| \pmb{\phi}_h - - \tilde{\phi} \right\|}
}
\end{equation}
where ${{\cal D}_{te}}=\{\pmb{\mu}_i, \pmb{\phi}_h(\pmb{\mu}_i)\}_{i=1}^{N_{{\cal D}_{te}}}$ is the test data set.

\item the average relative error of POD-G solutions
\begin{equation}
\varepsilon_{\bullet}
=\frac{1}{N_{{\cal D}_{te}}}
\sum\limits_{\pmb\mu ,\pmb{\phi}_h  \in {{\cal D}_{te}}}{
\frac{\left\| \pmb{\phi}_h- \tilde{\phi}-\mathbf{V}\pmb{\alpha}_{\bullet}\left( \pmb{\mu}\right)
\right\|}
{\left\| \pmb{\phi}_h - \tilde{\phi} \right\|}
}
\end{equation}
where the subscript "$\bullet$" is used to take the place of "POD-G","POD-NN","POD-PINN1" and "POD-PINN2". Thus ,$\pmb{\alpha}_{\bullet}\left( \pmb{\mu}\right)$ is the solution calculated from the responding method.
\end{enumerate}

Note that Projection, POD-G, PDNN, PINN and PRNN are all coded in Python. The networks built are are all under the framework of Pytorch. The training of networks are all  implemented on two GUPs (NVIDIA Quadro GP100 with) of a server class station. To solve the reduced-order model directly for POD-G, the solver "linalg.solve" in  Numpy library is employed. 


\subsection{Burgers' equation}
In this subsection, to validate the proposed methods, we consider the one-dimensional parameterized Burges' equation.
\begin{equation}
\left\{
\begin{aligned}
	 &\phi(x;\pmb{\mu}) \cdot \nabla \phi(x;\pmb{\mu}) -  \Delta \phi(x;\pmb{\mu})= s(x;\pmb{\mu})  \qquad -1 \le x \le 1\\
	 &\phi\left ( x= \pm1; \pmb {\mu} \right ) = 0
\end{aligned}\right.
\end{equation}
where $\pmb{\mu} =(\mu_1, \mu_2) \in [1,10] \times [1,10]$ and $s(x;\pmb{\mu})$ it the source term defined so that the exact solution is
\begin{equation}
	\phi(x; \pmb{\mu}) = (1+\mu_1 x)\sin(-\mu_2 x/3) (x^2-1) .
\end{equation}
The solution at the two end points are simply set zero to avoid the influence of boundary conditions. The problem is solvld with Chebyshev pseudospectral(PS) method with $N_p+1$=128+1 Chebyshev Gauss-Lobatto points.



We are are interested in the prediction accuracy of   Projection, POD-G, PDNN, PINN and PRNN. The accuracy of projection and POD-G is only determined by two factors, namely the sample size for building RB and and the number of chosen modes, given that the nonlinear solver for POD-G is accurate. Here we term the two factors as essential factors.
Besides the two factors, neural network like PDNN, PINN and PRNN are also determined by network architecture, training data size training method, and so on. We first study the essential factors. For building the RB, $N_{\rm{POD}}=320$ random parameter samples are independently generated with Sobol sequences, a quasi-random low-discrepancy sequences. Fig. ##, presents the singular value distribution for the sample number. It is shown that the singular value decays quickly, implying that only a small number of RB modes can represent the dynamic of the problem. Meanwhile, the sample number can also be largely reduced.
Thus, for comparison, $N_{\rm{POD}}=10$ and 80 random parameter samples are independently generated with Sobol sequences. By choosing the first $m, 1 \le m \le 10$ modes respectively, the five methods are applied to the problem after the RB are built for $N_{\rm{POD}}=10$, 80 and 320, respectively. To illustrate the prediction accuracy, a test data set of size $101/101=10201$ is generated on a tensor-product grid $\{1+ 0.09i\}_{i=0}^{i=100} \times \{1+ 0.09i\}_{i=0}^{i=100}$. The dense test grid will grantee the reliability of the prediction accuracy, which is shown in Fig. ##.  It is shown that the PINN and PRNN  can almost have a similar error distribution of first dropping down with $m$ and turning flat after  $m=5$. The error curve of the PINN and PRNN keeps flat almost at the same level. As both PINN and PRNN are not deterministic solvers, they can only make the loss drop down to a limited level, e.g. 1E-6 for this problem. As POD-G employs the fsolve solver in




\subsection{Lid-driven cavity}

\subsection{Rayleigh-B\'enard convection}


\section{Conclusions}


%% The Appendices part is started with the command \appendix;
%% appendix sections are then done as normal sections
%% \appendix

%% \section{}
%% \label{}

%% If you have bibdatabase file and want bibtex to generate the
%% bibitems, please use
%%
%%  \bibliographystyle{elsarticle-num}
%%  \bibliography{<your bibdatabase>}

%% else use the following coding to input the bibitems directly in the
%% TeX file.
\section*{Acknowledgments}
The first author is financially supported by Xi'an Jiaotong University xxx.

%% References
\bibliographystyle{elsarticle-num}
\bibliography{ref}

\end{document}
%%\endinput
%%
%% End of file `elsarticle-template-num.tex'.
